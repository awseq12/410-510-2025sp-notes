\documentclass{lecturenotes}

\usepackage[colorlinks,urlcolor=blue]{hyperref}
\usepackage{doi}


\title{Equality and Isomorphism}
\coursenumber{CSE 410/510}
\coursename{Programming Language Theory}
\lecturenumber{5}
\semester{Spring 2025}
\professor{Professor Andrew K. Hirsch}

\begin{document}
\maketitle
\section{Equality}
\label{sec:Equality}
\subsection*{Refl}
\begin{itemize}
    \item Refl is a constructor for equality, it proves that two things are equal judgementally
    (result in the same result).
    \newline Take for example: ~$2$ and ~$1 + 1$.
    \begin{itemize}
        \item refl states that these two things are equal since they would result in the same thing.
    \end{itemize}
    \item As refl proves equality, it holds that the terms are
    \begin{itemize}
        \item Reflexive: For every $x$, $x$ $R$ $x$
        \item Symmetrical: For every $x$ and $y$, if $x$ $R$ $y$ then $y$ $R$ $x$
        \item Transitive: For every $x$, $y$, and $z$, if $x$ $R$ $y$ and $y$ $R$ $z$ then $x$ $R$ $z$
    \end{itemize}
  \end{itemize}
\subsection*{Rewrite}
\begin{itemize}
    \item Allows us to rewrite the type of the goal 
    \begin{itemize}
        \item It effectively allows us to replace the goal with an already proven equality
        \item Rewrite also allows us to apply the commutative property to addition
    \end{itemize}
\end{itemize}
\subsection*{Leibniz equality}
\begin{itemize}
    \item Also Known as the identity of indiscernibles: two objects are equal if have all properties in common.
    \item Through the Leibniz equality, we can also disern that if two objects $x$ and $y$ are equal, then any
    \newline property of $y$ is also the same for $x$
    \begin{itemize}
        \item As an example Let's consider the following:
        \begin{center}
            $y \equiv x$ where: \newline
            $y = 3$ and \newline
            $x = 2 + 1$ \newline
            $y + 2 = 5$ \newline
        \end{center}
        through the Leibniz equality, we know that $x$ must have the same properties when we add $2$ to it, \newline
        and therefore we can disern that $x + 2 = 5$ 
    \end{itemize}
\end{itemize}

\section{Isomorphism}
\label{sec:Isomorphism}
\subsection*{Extensionality}
\begin{itemize}
    \item We can assert that two functions are identical if when applied to the same arguements, results in the same result.
\end{itemize}
\subsection*{Isomorphism}
\begin{itemize}
    \item Denoted by $A \cong B$, two sets are Isomorphic if we can satisfy the following:
    \begin{itemize}
        \item A function $x$ from set $A$ to set $B$
        \item A function $y$ from set $B$ to set $A$
        \item Evidence of identity between functions $x$ and $y$ (functions are inverse of each other)
    \end{itemize}
\end{itemize}
\end{document}

%%% Local Variables:
%%% mode: latex
%%% TeX-master: t
%%% TeX-engine: luatex
%%% End:
